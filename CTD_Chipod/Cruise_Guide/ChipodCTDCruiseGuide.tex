\documentclass[11pt]{article}
\usepackage{geometry}                % See geometry.pdf to learn the layout options. There are lots.
\geometry{letterpaper}                   % ... or a4paper or a5paper or ... 
%\geometry{landscape}                % Activate for for rotated page geometry
%\usepackage[parfill]{parskip}    % Activate to begin paragraphs with an empty line rather than an indent
\usepackage{graphicx}
\usepackage{amssymb}
\usepackage{amsmath}
\usepackage{epstopdf}
\usepackage{hyperref}
\usepackage{natbib}
\DeclareGraphicsRule{.tif}{png}{.png}{`convert #1 `dirname #1`/`basename #1 .tif`.png}



\graphicspath{
}

\title{CTD-Chipod Cruise Guide}
\author{Andy Pickering}
\date{}                                           % Activate to display a given date or no date



\begin{document}
\maketitle

\tableofcontents
\newpage


%~~~~~~~~~~~~~~~~~~~~~~
\section{Introduction}

This guide is for those tending CTD-chipods on research cruises. It contains basic information about the instruments, how to monitor and download the data, and common issues that arise during these cruises. 

%~~~~~~~~~~~~~~~~~~~~~~
\section{Chipods}

Basic info on chipods. Chipods are instrument packages developed by the OSU Ocean Mixing Group to measure turbulence. They consist of a pressure case, logger board, accelerometer, and thermistor(s). The thermistors measure fluctuations in temperature. Combined with CTD data, turbulent dissipation rate and diffusivity can then be estimated. 

Big vs mini. 

Example Pictures.

\subsection{Mounting on CTD Rosette}

Chipods should be mounted on the CTD rosette in a way that the thermistors will most closely sample `clean' water (not contaminated by the wake of the CTD). Typically, several sensors are deployed facing both up (`uplookers') and down (`downlookers'); the down(up)lookers will see `clean' water on the down(up) casts. Uplooking sensors can be mounted such that they extend beyond the top of the Rosette. Downlookers cannot extend below since the Rosette is placed on deck between casts, but should be mounted as close as possible to the bottom. 

%~~~~~~~~~~~~~~~~~~~~~~
\section{Data Download}

Connect via...

Data is downloaded using software xxx..


%~~~~~~~~~~~~~~~~~~~~~~
\section{Naming and File Conventions}

Because chipods are deployed on many different cruises and by different people, it is important that some common conventions are used. 

Filename has timestamp in it (is this automatic in software, or does it need to be named manually?).

Suffix is the logger SN.

Example:

Chipod data go in folders named by SNs:
\verb+/[Cruise Name]/CruiseData/Chipod/[SN]+

CTD data should go in:
\verb+/[Cruise Name]/CruiseData/CTD/+

%~~~~~~~~~~~~~~~~~~~~~~
\section{Troubleshooting}

Some common issues:

\subsection{Can't communicate with instrument}
connect directly to instrument with the shortest USB cable you have, try a second one, then open P-case and connect directly into board, 

\subsection{Can't download data}

\subsection{Data looks like a flat line}
Sensor not connected or shorted?


%~~~~~~~~~~~~~~~~~~~~~~
\section{Logs/Notes}

Use provided log sheets.

Be sure to note any issues encountered and/or changes to chipod configuration. 


%~~~~~~~~~~~~~~~~~~~~~~
\section{Analysis}

Plot raw data and check for obvious issues (every 3 days?)

If possible, email png figures of raw data. 

\end{document}  
