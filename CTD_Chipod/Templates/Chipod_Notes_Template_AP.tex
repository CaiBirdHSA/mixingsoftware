%~~~~~~~~~~~~~~~~~~~~~~~~~~~~~~~~~~~~~~~~~~~~~~~~~~~~~~~
%
% This is a template for standard notes for a CTD-chipod cruise/deployment. 
%
% - Use command-F to find 'XXXX' and replace with the cruise name.
% - Edit graphics paths. Add folder for each chipod deployed.
% - Make table w/  MakeTableChiDeploy.m and paste in appropriate spot.
% -  Make table w/  SummarizeProc_XXXX.m  and paste in appropriate spot.
%
% 06/16/16 - A.Pickering - apickering@coas.oregonstate.edu
%~~~~~~~~~~~~~~~~~~~~~~~~~~~~~~~~~~~~~~~~~~~~~~~~~~~~~~~


%~~~~~~~~ Set up document (these are all just defaults I think)
\documentclass[11pt]{article}
\usepackage{geometry}                % See geometry.pdf to learn the layout options. There are lots.
\geometry{letterpaper}                  % ... or a4paper or a5paper or ... 
%\geometry{landscape}                % Activate for for rotated page geometry
%\usepackage[parfill]{parskip}      % Activate to begin paragraphs with an empty line rather than an indent
\usepackage{graphicx}
\usepackage{amssymb}
\usepackage{amsmath}
\usepackage{epstopdf}
\usepackage{hyperref}
\usepackage{natbib}
\DeclareGraphicsRule{.tif}{png}{.png}{`convert #1 `dirname #1`/`basename #1 .tif`.png}
%~~~~~~~~


%*** Set graphics path for figures
\graphicspath{
{/Users/Andy/Cruises_Research/Chipod/XXXX/}
{/Users/Andy/Cruises_Research/Chipod/XXXX/Figures/}
{/Users/Andy/Cruises_Research/ChiPod/XXXX/Data/proc/Chipod/SNyyyy/figures/}
}

\title{XXXX CTD-Chipod Notes}
\author{Andy Pickering}
%\date{}                                           % Activate to display a given date or no date


\begin{document}
\maketitle

\tableofcontents
\newpage

%~~~~~~~~~~~~~~~~~~~~~~~~~~~~~
\section{About}

Notes on processing and analysis of CTD-chipod data collected during XXXX cruise..

\begin{figure}[htbp]
\includegraphics[width=38pc]{XXXX_kml_map.jpg}
\caption{Map of CTD cast locations during cruise XXXX.}
\label{map}
\end{figure}


%~~~~~~~~~~~~~~~~~~~~~~~~~~~~~
\section{Data and Processing}

Data paths are set w/ : 
\begin{itemize}
\item \verb+Load_chipod_paths_XXXX.m+
\end{itemize}

Chipod deployment info is given in 
\begin{itemize}
\item \verb+Chipod_Deploy_Info_XXXX+
\end{itemize}


%~~~~~~~~
\subsection{CTD}

Raw (hex) CTD data are processed with:
\begin{itemize}
\item \verb+Process_CTD_hex_XXXX.m+
\end{itemize}

%~~~~~~~~
\subsection{Chipod}

% **** Made w/ MakeTableChiDeploy
\begin{table}[htdp]
\caption{Summary of Chipod deployment .}
\begin{center}
\begin{tabular}{|c|c|c|c|}
\hline
\end{tabular}
\end{center}
\label{default}
\end{table}%


Raw chipod files are plotted with 
\begin{itemize}
\item \verb+PlotChipodDataRaw_XXXX.m+
\end{itemize}
to check for obvious issues/malfunctions with any of the instruments; these plots are saved in \verb+/Figures/chipodraw/+. 

%*** Note any major issues here
Based on a quick look at these:
\begin{itemize}
\item
\end{itemize}

Files with obviously bad or missing data (based on above raw plots) are noted in \verb+bad_file_list_XXXX.m+ and, which will prevent them from being loaded in the processing.

Processing scripts are:
\begin{itemize}
\item \verb+MakeCasts_CTDchipod_XXXX.m+
\item \verb+DoChiCalc_XXXX.m+
\end{itemize}

%~~~~~~~~
\subsection{Further processing and analysis}

\begin{itemize}
\item Data from all casts are combined into a single structure w/ \verb+MakeCombinedStruct_XXXX.m+
\item 
\end{itemize}


\newpage
%~~~~~~~~~~~~~~~~~~~~~~~~~~~~~
\section{Processing Notes}

\begin{itemize}
\item Table \ref{chidepinfo} gives the info for $\chi$pods deployed.
\item Table \ref{procsum} gives a summary of the processing from MakeCasts.
\end{itemize}

% **** table made in SummarizeProc_XXXX.m, copy latex here
\begin{table}[htdp]
\caption{Some $\chi$pod processing summary info for XXXX. }
\begin{center}
\begin{tabular}{|c|c|c|c|}
\hline
SN & $\chi$ data & T1cal Good & toffset $<$ 1min \\ 
\hline
\hline
\hline
\end{tabular}
\end{center}
\label{procsum}
\end{table}


\begin{figure}[htbp]
\includegraphics[scale=0.7]{XXXX_haveChiData_all.png}
\caption{Figure showing which casts there is $\chi$pod data for. Note castid is NOT necessarily same as cast numbers in CTD files.}
\label{ischidata}
\end{figure}


%~~
\subsubsection{Time Offsets}


\begin{figure}[htbp]
\includegraphics[scale=1]{XXXX_timeoffsets_all.png}
\caption{Time-offsets for all $\chi$pods, found by aligning with CTD data.}
\label{toffs}
\end{figure}



\newpage
%~~~~~~~~~~~~~~~~~~~~~~~~~~~~~
\section{Example Raw Chipod Data for 1 cast}

\begin{figure}[htbp]
\includegraphics[scale=0.7]{SNyyyy_001_Fig1_RawChipodTS.png}
\caption{Raw chipod data from SNyyyy for a CTD cast.}
\label{snyyyy_1}
\end{figure}



\newpage
%~~~~~~~~~~~~~~~~~~~~~~~~~~~~~
\section{Results}


\clearpage
%~~
\subsubsection{CTD Data }

% Figure showing pcolor of t and s from all casts
\begin{figure}[htbp]
\includegraphics[scale=0.7]{XXXX_ctd_t_s.png}
\caption{Plot of temperature and salinity from CTD downcasts on all casts.}
\label{}
\end{figure}


\clearpage
%~~
\subsubsection{Data from each individual Chipod}


% *** One for each chipod
\begin{figure}[htbp]
\includegraphics[scale=0.7]{XC_SNX_xx_Vs_lat_AllVars.png}
\caption{All chipod profiles from sensor SNX. Variables are: N2, dTdz, chi, eps, and KT.}
\label{}
\end{figure}



\clearpage
\newpage
%~~
\subsubsection{One variable from all Chipods}

\begin{figure}[htbp]
\includegraphics[scale=0.7]{XXXX_chi_AllSNs_Vslat.png}
\caption{Plot of one variable from all chipods.}
\label{}
\end{figure}

\begin{figure}[htbp]
\includegraphics[scale=0.7]{XXXX_KT_AllSNs_Vslat.png}
\caption{Plot of one variable from all chipods.}
\label{}
\end{figure}



\clearpage
\newpage
%~~
\subsubsection{One variable from all Chipods with bad casts screened out}


\begin{figure}[htbp]
\includegraphics[scale=0.7]{XXXX_Transect_AllSN_chi_screened.png}
\caption{Plot of one variable from all chipods, with bad casts eliminated.}
\label{}
\end{figure}

\begin{figure}[htbp]
\includegraphics[scale=0.7]{XXXX_Transect_AllSN_eps_screened.png}
\caption{Plot of one variable from all chipods, with bad casts eliminated.}
\label{}
\end{figure}


\begin{figure}[htbp]
\includegraphics[scale=0.7]{XXXX_Transect_AllSN_KT_screened.png}
\caption{Plot of one variable from all chipods, with bad casts eliminated.}
\label{}
\end{figure}



\clearpage
\newpage
%~~
\subsubsection{Best Combined Transect}


\begin{figure}[htbp]
\includegraphics[scale=0.7]{XXXX_Transect_allSN_BestCombo.png}
\caption{Best combined transect from all chipods. Bottom panel indicates which sensor is used for each cast.}
\label{}
\end{figure}




\clearpage
\newpage
%~~~~~~~~~~~~~~~~~~~~~~~~~~~~~
\section{Scatter Plots}

\begin{figure}[htbp]
\includegraphics[scale=0.7]{XXXX_ChiEpsKtVsdTdz.png}
\caption{Scatter plot of chipod variables vs dTdz .}
\label{}
\end{figure}



\newpage
\clearpage
\newpage
%~~~~~~~~~~~~~~~~~~~~~~~~~~~~~
\section{To-Do}

\begin{itemize}
\item 
\end{itemize}


\end{document}  

%~~~~~~~~~~~~~~~~~~~~~~~~~~~~~~~~~~~~~~~~~~~~~~~~~~~~~~~~~~~~~~
%~~~~~~~~~~~~~~~~~~~~~~~~~~~~~~~~~~~~~~~~~~~~~~~~~~~~~~~~~~~~~~

